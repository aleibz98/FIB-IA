\documentclass[a4paper, 12pt, UTF8]{article}

\usepackage[dvipsnames]{xcolor} % Code highlighting color
\usepackage[spanish]{babel} % Language 
\usepackage{indentfirst}
\usepackage{fontspec} 
\usepackage{fullpage}
\usepackage{csquotes}
\usepackage[a4paper, margin=2cm]{geometry} % To change the margins
\usepackage{graphicx} % Insert images
\usepackage[hidelinks]{hyperref} % Links color
\usepackage[final]{pdfpages}
\usepackage{ragged2e}
\usepackage{wrapfig} %To Text wrap
\usepackage{listings} % Add code
\usepackage{verbatim}
\usepackage[backend=bibtex,citestyle=ieee]{biblatex}
\usepackage{nameref}
\usepackage{tikz}
\usepackage{float}
\usepackage{subcaption}
\usepackage{url}
\usepackage{multirow}
\usepackage{colortbl}

\setlength{\parskip}{0.7em}
%\setlength{\parindent}{1cm}
\linespread{1.25}

\title{
	\Huge
	\textbf{Trabajo innovación} \\
	\scshape Uso de redes neuronales para la clasificación de células
	}
\author{
	Marc Asenjo i Ponce de León \and
	Joan Marcè i Igual \and
	Iñigo Moreno i Caireta
	}
\date{\today}

\addbibresource{Bibliografia.bib}


\begin{document}

\maketitle

\begin{figure}
	\centering
	\includegraphics[width=\linewidth]{./simple_FIB}
\end{figure}

\newpage
\tableofcontents

\newpage

\section{Introducción}

En este trabajo nos hemos centrado en la investigación\cite{deepLearning} sobre distintos métodos de clasificación de células, donde se ha desarrollado un nuevo método que no necesita que sean marcadas. Con esta nueva técnica és posible una clasificación más precisa de las células de un organismo permitiendo así avances en investigaciones como la del cáncer o el biodiésel.

Para poder analizar todos los datos que generan en los procesos de observación. Se utilizan distintos algoritmos entre los cuales están las redes neuronales.


\section{Sistemas de clasificación celular}
Los sistemas de clasificación celular son métodos para detectar ciertos atributos de las células que se encuentran en un conjunto y poder diferenciarlas de acuerdo con los valores de estos atributos. Estos sistemas tienen muchas y muy variadas aplicaciones, algunos ejemplos típicos y muy utilizados hoy en día serian la detección de células cancerígenas, detección de parásitos en tejidos celulares u obtener la cantidad de un componente que tiene una célula para evaluar su validez para un uso en concreto, como es el caso de las células de un alga para conocer su potencial como biocombustible. Sin embargo, de todos sus usos, sin lugar a dudas los más utilizados son los de aspecto médico, así que nos centraremos en ejemplos de éstos.

La mayoría de las técnicas utilizadas están relacionadas con las biotecnologías y son relativamente nuevas, así que se puede considerar un campo en desarrollo y en el que se hacen avances constantemente.

\section{Ejemplos actuales}
Actualmente no hay muchos sistemas de clasificación celular generales, es decir que puedan ser utilizados para muchas aplicaciones distintas, sino que la mayoría son muy especializados o sólo son válidos para determinar una propiedad específica de las células que están en estudio. Algunos de estos sistemas usan colorantes, otros identificadores de otros tipos, y también hay los que sirven para obtener imágenes que puedan ser analizadas. Empezaremos con el primer caso.

Hay varios tratamientos que nos sirven para sacar información de un tejido celular a base de colorantes. Sin embargo, a parte de los más básicos, hay dos que destacan: la histoquímica y citoquímica. Ambas técnicas sirven para detectar la presencia de moléculas que no serían visibles con colorantes generales, revelándolas usando una reacción química. Se modifica el tejido celular para luego poder colorearlo, o se hace una reacción que involucra las moléculas que se quieren detectar, produciendo una sustancia con color. Un caso específico es la llamada Immunohistoquímica , en la cual se usan anticuerpos con colorante, que se enganchan a membranas celulares específicas. En todos estos casos, sin embargo, tenemos los mismos problemas: necesitamos una preparación previa del tejido para asegurarnos que no se modifican elementos de las células con la reacción química que se quiere provocar, y además necesitamos agentes distintos para detectar moléculas distintas, haciendo estas pruebas válidas solamente para casos muy específicos. 

Después tenemos técnicas muy simples pero efectivas en casos particulares, como es el caso del fraccionamiento celular. Ésta técnica se centra en la diferenciación de distintas partículas, generalmente células, con densidad o tamaño distintos. Esto se consigue con una centrifugación a una velocidad creciente, después de la cual la muestra inicial queda dividida: los elementos más grandes y densos pasan más rápidamente al fondo del recipiente centrifugado que los más pequeños y poco densos. A esta técnica en concreto se le llama separación por centrifugación diferencial. Como podemos observar, sólo nos sirve para casos muy concretos y para la diferenciación de una sola propiedad de células.

Como se ha comentado, existen otros sistemas de detección que no se basan en colorantes. Un ejemplo de un método bastante nuevo y todavía en desarrollo es el basado en nanopartículas de oro para la detección de células cancerígenas. Éstas nanopartículas interaccionan con las membranas celulares, permitiendo la diferenciación entre las que son cancerígenas y las que no. Éste sistema tiene la ventaja que no es tan agresivo como los sistemas de coloración que hemos visto, y puede ser aplicado a una persona, en vez de a un tejido extraído. También es superior a muchos tratamientos actuales debido a que es altamente específico, es decir que no afecta a las células sanas, provocando así efectos secundarios al tratamiento. Una vez conseguida esta detección, se podrían utilizar las nanopartículas para varias cosas, como destruir las células con las que han interaccionado, usando una señal láser. Sin embargo, hay que tener en cuenta que todo esto se mueve dentro de un margen de error, es decir que puede fallar y que se marquen células sanas, por lo que un remedio como destruirlas podría ser contraproducente.

Sin embargo, los métodos más fácilmente comparables con la técnica que queremos estudiar en este trabajo son aquellos que también usan imágenes. Para obtener estas imágenes existen muchos sistemas muy usados, como las ecografías o la resonancia magnética, que con ultrasonido o con ondas de radio potentes respectivamente nos sirven para obtener una imagen de tejidos celulares que sirven para su análisis posteriormente. Otro ejemplo seria el PET-TC, que utiliza pequeñas cantidades de materiales radioactivos llamados radiosondas, una cámara especial y una computadora para ayudar a evaluar las funciones de sus órganos y tejidos. Estos materiales enganchan una substancia que podremos ver a las células que los consumen, y así se pueden identificar aquellas células con mayor actividad en un aspecto en concreto. En el caso específico de las células cancerígenas, éstas son las que hacen la mitosis de forma más activa. Todos estos ejemplos, sin embargo, sirven solo para evaluar conjuntos de células en conjunto.

\begin{figure}[H]
	\centering
	\includegraphics[width=\linewidth]{intro_1}
	\caption{Resultados obtenidos en un PET-TC en el que se aprecian las zonas de máxima actividad celular}
\end{figure}

Sin embargo, una vez obtenida la imagen, no tenemos sistemas que de forma automática detecten los problemas visibles en ellas, y tampoco tenemos muchas formas de ver esta información a nivel celular. Normalmente se usa el conocimiento de una persona para la evaluación de las imágenes, y es por esto que el uso de un método de inteligencia artificial además de una visualización a nivel celular muy rápida es la solución perfecta para la diferenciación o clasificación de células, que es exactamente lo que veremos a continuación.

\section{Introducción al sistema de clasificación}

\subsection{Adquisición de datos}
El sistema que han ideado se basa en combinar tecnologías de fotometría e inteligencia artificial para clasificar correctamente una célula. El sistema tiene que ser capaz de analizar muchas células por segundo mientras pasan por un canal, de esta manera se podrán clasificar las células, aunque haya muchas pocas ocurrencias de una de las opciones, como pasa en el caso de la detección de células cancerígenas, donde la cantidad de células cancerígenas es mucho inferior a la de células en buen estado si la enfermedad aún no ha llegado a los estados más destructivos. Para ello han desarrollado un sistema innovador con el que consiguen capturar unos 36 millones de imágenes por segundo, con lo que se consigue una capacidad de analizar 100.000 células por segundo. Además permite que las células que viajan por el nanotubo se puedan desplazar a 10 metros por segundo.

\begin{figure}[H]
	\centering
	\includegraphics[width=0.95\textwidth]{Phase}
	\caption{Ejemplo de las fotos obtenidas por el sistema}
	\label{fig:phase}
\end{figure}

\subsection{Datos calculados}
El sistema es capaz de calcular muchos datos de cada célula, pero se pueden clasificar en dos grupos: los relacionados con la morfología de la célula y los que resultan de los análisis fotométricos. Entre los de la morfología de la célula se encuentran: el diámetro, el área, la circularidad de la célula, etc. El análisis fotométrico aporta datos como el índice de refracción de las células, su absorción y dispersión de luz, etc. Se puede encontrar una lista de todos los datos que utilizan en el informe científico. Como tienen tantos datos, muchos están relacionados entre ellos (ver \hyperref[fig:features]{figura~\ref{fig:features}}), haciendo un análisis de la correlación entre los datos, han encontrado que hay tres grupos muy relacionados: los datos relacionados con la morfología, los relacionados con la refracción, y los relacionados con la perdida y dispersión de la luz.

\begin{figure}[H]
	\centering
	\includegraphics[width=0.95\textwidth]{features}
	\caption{(a) Correlación entre los diferentes datos calculados
		(b) Fiabilidad de cada uno de los paràmetros para hacer la clasificación}
	\label{fig:features}
\end{figure}

\section{Uso de tecnologías de inteligencia artificial}

Con toda esta gran cantidad de datos, se usa un algoritmo de redes neuronales para poder clasificar las células. El algoritmo de redes neuronales se basa en un árbol dirigido de nodos donde hay diferentes capas de nodos (ver \hyperref[fig:neural]{figura~\ref{fig:neural}}). Cada nodo hace una media ponderada de las salidas de los nodos de la capa anterior y después le aplica una función h a esta media para determinar la salida. La función h varía dependiendo de la capa, principalmente se usan dos funciones: la “función sigmoidea logística” es $h(a)=\frac{1}{1+e^{-a}}$ y la “unidad lineal rectificada” es $h(a)=max(0,a)$. Además, a cada capa se le añade un nodo que su salida siempre es uno. 
	
La última capa de nodos es la capa de decisión, en ella, cada nodo representa una de las posibles clasificaciones de la red. El algoritmo dirá que una célula pertenece a una clasificación dependiendo de un parámetro de aceptación.

\begin{figure}[h!]
	\centering
	\includegraphics[width=0.95\textwidth]{neural}
	\caption{Esquema de la red neural}
	\label{fig:neural}
\end{figure}

Para encontrar los parámetros de la ponderación de las entradas de cada nodo se usa un método iterativo. Inicialmente, se asignan de manera aleatoria \cite{training}. En cada iteración se le pasa a la red neural una cantidad de ejemplos de los cuales se sabe el resultado esperado. Para cada uno de los nodos de la capa de decisión se calcula la curva ROC \cite{roc} (del inglés Receiver Operating Characteristic). Esta curva es una representación de la sensibilidad  (porcentaje de clasificación correcta cuando el resultado tendría que ser positivo \cite{sensibilidad}) y especificidad (porcentaje de clasificación correcta cuando el resultado tendría que ser negativo). La curva se construye calculando la sensibilidad y especificidad para cada valor del parámetro de aceptación. Una vez hemos encontrado la curva ROC de cada nodo de la capa de decisión,  se ajustan los nodos de tal manera que el área que queda por debajo de cada curva ROC sea lo más alta posible.

\section{Impacto del descubrimiento}

Esta investigación permite hacer operaciones más profundas ya que normalmente el impacto que puede ocasionar la clasificación clásica puede generar que se activen o se inhiban ciertas señales, cambiando el comportamiento de ciertos tipos de células. Así pues este estudio permite sobrepasar estos problemas ya que no es necesario marcar las células para investigarlas y se puede hacer con una gran precisión y velocidad.

Como con este análisis se puede obtener más precisión, sensibilidad y seguridad, se está empezando a trabajar en la clasificación de células en busca de las que son potencialmente cancerígenas. En la \hyperref[fig:impacto_1]{Figura~\ref{fig:impacto_1}} se observa como este tipo de análisis permite obtener todo un conjunto de características para cada célula individualmente. 
Así pues, es relativamente fácil, a partir de ciertos parámetros, adivinar si hay células cancerígenas dentro del organismo y saber cuáles son.

\begin{figure}[H]
	\centering
	\includegraphics[width=0.95\textwidth]{impacto_1}
	\caption{Ejemplo de la clasificación de células cancerígenas donde se han ordenado en función de su tamaño, concentración de proteínas y atenuación}
	\label{fig:impacto_1}
\end{figure}

Otro ejemplo de uso son los biofueles generados a partir de microorganismos. Estos convierten el dióxido de carbono en lípidos que son mejores que los obtenidos por la agricultura tradicional. 

Actualmente, en todo el mundo se esta intentando mejorar la productividad de estos microorganismos y con este método podrán ser capaces de seleccionar directamente las micro algas que tienen un factor de crecimiento mayor y esto, es esencial en la industria de producción de biofuel para poder generar grandes cantidades de este producto y rebajar los costes. Se puede ver en la \hyperref[fig:impacto_2]{Figura \ref{fig:impacto_2}} como se han podido clasificar las diferentes algas en función de los contenidos de lípidos.

\begin{figure}[H]
	\centering
	\includegraphics[width=\linewidth]{impacto_2}
	\caption{Ejemplo de la clasificación de micro algas, separando las que tienen un alto contenido de lípidos de las que tienen uno bajo}
	\label{fig:impacto_2}
\end{figure}

Así pues se puede ver que con este sistema se obtiene muchísima más precisión que con los métodos anteriores de manera que abre las puertas a un nuevo campo donde cada microorganismo se pueda clasificar individualmente. Se esta empezando a trabajar también en los microbots \cite{microbots} y quizá en un futuro se puedan emplear para clasificar todo aquello que no debería estar dentro de nuestro organismo como por ejemplo las bacterias o toda esa grasa innecesaria. 

\printbibliography

\end{document}