\documentclass[a4paper, 12pt]{article}

\usepackage[dvipsnames]{xcolor} % Code highlighting color
\usepackage[catalan]{babel} % Language 
\usepackage{fontspec} 
\usepackage{fullpage}
\usepackage[a4paper, margin=2cm]{geometry} % To change the margins
\usepackage{graphicx} % Insert images
\usepackage[hidelinks]{hyperref} % Links color
\usepackage[final]{pdfpages}
\usepackage{ragged2e}
\usepackage{wrapfig} %To Text wrap
\usepackage{listings} % Add code
\usepackage{verbatim}
\usepackage{tikz}
\usepackage{subcaption}
\usepackage{multirow}
\usepackage{colortbl}

\begin{document}

\title{\textsc{Sistema Basat en el Coneixement}}
\author{Marc Asenjo i Ponce de León \and
	Joan Marcè i Igual \and
	Iñigo Moreno i Caireta}
\date{\today}
\maketitle

\section{Identificació del problema}

\subsection{El problema}

Amb el creixent nombre de persones que adquireixen hàbits de vida saludable evitant el sedentarisme, la cadena de gimnasos "I'm no couch potato" vol desenvolupar un Sistema Basat en el Coneixement capaç de recomanar programes d'entrenament als seus futurs clients. Dins del procés de recomanació, el primer pas és establir les condicions físiques generals de la persona. Per això, es necessita primer una sèrie de dades bàsiques com el pes i la altura (a partir d'aquestes es pot calcular l'índex de massa corporal), l'edat i la pressió sanguínia (màxima i mínima). També es pot so\l.licitar que es realitzin exercicis senzills per tal d'obtenir alguns paràmetres com les pulsacions per minut, la sensació de cansament / mareig o la tibantor muscular després d'un minut de carrera sostinguda o de pujar un seguit de trams d'escala a ritme normal. 

A més, es necessitarà informació sobre els hàbits personals que donin una idea dels tipus d'activitats que realitza, com per exemple totes les activitats físiques que fa a la feina (estar assegut, de peu, moviments repetitius, aixecament de pes, esforços musculars, ...), activitats fora de la feina (estàtiques (televisió, lectura, ...), tasques domèstiques (posar una rentadora, planxar, ...), desplaçaments (anar a comprar a peu, passejar, ...)). D'aquestes activitats ens interessa la seva freqüència i la duració. Realitzar poca o molta activitat física a les activitats diàries ens donarà una idea de la intensitat inicial que l'usuari pot suportar i quina part dels objectius ja cobreixen aquestes. 

També es vol obtenir informació sobre la seva salut, com per exemple problemes múscul-esquelètics (mal d'esquena, articulacions, cervicals...) o dieta (consum de fruita, abús de sal, picar entre hores...). 
A part d'aquesta informació es serà necessari saber els objectius del programa que s'ha de crear (manteniment, posar-se en forma, rebaixar pes, musculació, flexibilitat, equilibri...) així com el temps diari disponible per l'entrenament (com a mínim 30 minuts diaris).
 
l sistema té un conjunt d'exercicis que es poden realitzar en un gimnàs preparats pels diferents objectius que puguin interessar a l'usuari (cada exercici pot tenir diferents objectius), com exercicis amb màquines (bicicleta estàtica, cinta de córrer, rem, stepper, pesos...), exercicis amb o sense peses pels diferents grups musculars, exercicis de terra o estiraments.
Aquests, tenen un conjunt de característiques i restriccions com per exemple el nombre de calories que es cremen per quantitat de temps, duració (màxima i mínima), nombre de repeticions (màximes i mínimes), els grups musculars que treballen, si estan contraindicats per alguna condició de l'usuari (pressió alta, problemes musculars o a les articulacions...), si no estan indicats per certes edats o si estan especialment pensats per alleugerar algunes condicions (mal d'esquena, mobilitat limitada...). També tenim informació sobre els exercicis que combinen millor amb cada exercici. La dificultat d'aquests (moderada, normal o difícil) pot estar lligada al propi exercici, al nombre de repeticions que es realitzen o a la condició física de l'usuari (fer 5 abdominals pot ser fàcil per a un usuari amb una condició física normal i sense sobrepès però difícil per a algú amb sobrepès).

El sistema ha de generar un program d'entrenament per a una setmana (de coma a mínim 30 minuts diaris), creant per a cada dia una seqüència d'exercicis adequada al temps del que es disposa. Els exercicis han d'escollir-se segons les condicions físiques/mèdiques de l'usuari. Aquests exercicis han d'anar encaminats principalment a l'objectiu que ha indicat l'usuari atenent a les condicions de partida de l'usuari. Hi ha d'haver suficients exercicis diferents a cada sessió i durant la setmana per evitar la monotonia. 
\end{document}