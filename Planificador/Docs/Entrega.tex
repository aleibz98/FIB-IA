\documentclass[a4paper,12pt, UTF-8]{article}

\usepackage[dvipsnames]{xcolor} % Code highlighting color
\usepackage[catalan]{babel} % Language 
\usepackage{indentfirst}
\usepackage{fontspec} 
\usepackage{fullpage}
\usepackage[a4paper, margin=2cm]{geometry} % To change the margins
\usepackage{graphicx} % Insert images
\usepackage[hidelinks]{hyperref} % Links color
\usepackage[final]{pdfpages}
\usepackage{ragged2e}
\usepackage{wrapfig} %To Text wrap
\usepackage{listings} % Add code
\usepackage{verbatim}
\usepackage{nameref}
\usepackage{tikz}
\usepackage{subcaption}
\usepackage{multirow}
\usepackage{colortbl}
\usepackage{listings}

\setlength{\parskip}{0.7em}
%\setlength{\parindent}{1cm}
\linespread{1.2}

\title{
	\Huge
	\textbf{Pràctica 3} \\ 
	\scshape Planificador
}
\author{
	Marc Asenjo i Ponce de León \and
	Joan Marcè i Igual \and
	Iñigo Moreno i Caireta
}
\date{\today}

\begin{document}
	
\lstdefinelanguage{pddl}
{
	keywords={
		define,
	},
	morekeywords={[2]
		domain,
		problem,
		requirements,
		predicates,
		types,
		objects,
		action,
		init,
		goal,
	},
	morekeywords={[3]
		parameters,
		vars,
		precondition,
		effect,
	},
	morekeywords={[4]
		forall,
		or,
		and,
		not,
		when,
		=,
		exists,
		imply,
	},
	comment=[l]{\;},
	sensitive=true
}[keywords, comments]

\lstdefinelanguage[mapl]{pddl}[]{pddl}
{
	morekeywords={[2]
		sensor,
	},
	morekeywords={[3]
		sense,
		replan,
	},
	morekeywords={[4]
		KIF,
	},
}

\lstset{language=pddl}



\maketitle

\begin{figure}
	\centering
	\includegraphics[width=0.8\linewidth]{./simple_FIB}
\end{figure}

\newpage

\tableofcontents

\newpage

\section{Introducció}

\subsection{El problema}

Aquest problema és una continuació de la pràctica 2 realitzada. El client després de quedar satisfet amb el nostre sistema de recomanació d'exercicis construït en CLIPS ens demana ara una eina senzilla que, basant-se en exercicis que l'usuari està realitzant en aquest moment i el seu nivell de dificultat (de l'1 al 10), li faci un pla d'entrenament físic quinzenal amb l'objectiu d'aconseguir pujar de nivell un conjunt d'exercicis escollits. El pla d'entrenament ha de tenir en compte que no es pot realitzar un exercici amb nivell $X$ de dificultat si no s'ha realitzat anteriorment el mateix exercici amb dificultat $X - 1$ i que hi ha exercicis que s'han de fer abans que altres. Es defineixen les següents relacions entre exercicis:

\begin{description}
	\item[Exercicis precursors:] un exercici és precursor d'un altre si s'ha de fer abans que l'altre exercici el mateix dia d'entrenament, sense haver-hi cap exercici intercalat entre ells. És una restricció estricta que no es pot trencar.
	\item[Exercicis preparadors:] són exercicis que s'han de fer abans d'un cert exercici el mateix dia d'entrenament però no han d'anar immediatament abans (un exemple són els estiraments, que preparen diversos exercicis per a les sessions d'entrenament). És una restricció estricta que no es pot trencar.
	\item[Exercicis nivell $N + 1$:] serveixen per modelar l'evolució del nivell de dificultat d'un exercici segons passen els dies. Per cada exercici es pot pujar un nivell $N + 1$ només si aquest exercici s'ha realitzat anteriorment al nivell $N$.
\end{description}

Així doncs al sistema se li ha de donar coneixement sobre:

\begin{itemize}
	\item els exercicis que es poden fer al gimnàs
	\item els exercicis predecessors d'un exercici
	\item els exercicis preparadors d'un exercici
	\item els exercicis que l'usuari ja està fent i el nivell al que els fa
	\item els nivells incrementats d'alguns exercicis als que es vol arribar al final de la quinzena.
\end{itemize}

El resultat és un pla d'entrenament que:
\begin{itemize}
	\item reflexa els exercicis mínims que l'usuari ha de realitzar per arribar als nivells objectiu (\emph{important:} el planificador no ha d'omplir el pla amb exercicis extra, només els necessaris per arribar al dia 15 al nivell desitjat. Si sobren espais es sobreentén que se li dona a l'usuari la llibertat per fer altres exercicis suplementaris al pla d'entrenament).
	\item per a cada exercici al pla, indica el dia a fer-lo i a quin nivell per arribar als objectius quinzenals.
	\item per tots els exercicis del pla es compleix en tot moment que els seus predecessors es fan just abans el mateix dia, sense cap exercici intercalat pel mig. 
	\item per tots els exercicis del pla es compleix en tot moment que els seus exercicis preparadors es fan abans el mateix dia (però hi pot haver exercicis intercalats).
	\item per tot exercici del pla amb nivell $N$ es compleix en tot moment que el mateix exercici ha estat realitzat algun dia anterior amb nivell $N$ o $N - 1$.
\end{itemize}

\subsection{Extensions del problema}

Aquest problema consta de diverses extensions per poder obtenir millor nota, les extensions són les següents:

\begin{description}
	\item[Problema bàsic:] Al pla d'entrenament tots els exercicis tenen 0 o 1 preparadors i cap predecessor. El planificador és capaç de trobar un pla per poder realitzar els exercicis al nivell final de la quinzena encadenant exercicis i pujant el nivell si fa falta segons van passant els dies, tenint en compte que cada exercici te un o cap exercici preparador (els estiraments no tenen cap preparador) i que per tot exercici del pla amb nivell $N$ es compleix que en tot moment en el que s'ha realitzat l'exercici anteriorment s'ha fet amb nivell $N$ o $N - 1$.
	\item[Extensió 1:] Els exercicis poden tenir de 0 a N preparadors però cap predecessor. El planificador és capaç de construir un pla per poder realitzar els objectius al nivell desitjat encadenant exercicis i pujant de nivell si fa falta segons van passant els dies, tenint en compte que per a tot exercici que pertany al pla, tots els seus exercicis preparadors pertanyen al pla i estan abans el mateix dia.
	\item[Extensió 2:] Extensió 1 + 	els exercicis 0 o 1 exercici predecessor. El planificador és capaç de construir un pla per poder realitzar els exercicis al nivell objectiu encadenant exercicis i pujant de nivell si fa falta segons els dies, on per a tot exercici que pertany al pla, tots els seus exercicis precursors i preparadors estan co\l.locats el mateix dia i per a tot exercici del pla amb un nivell $N$ es compleix en tot moment que el mateix exercici ha estat realitzat anteriorment amb un nivell $N + 1$
	\item[Extensió 3:] El planificador controla que no hi hagi més de 6 exercicis al dia.
	\item[Extensió 4:] Els exercicis tenen assignat el temps (en minuts) que es triga a realitzar-los (incloent el temps de descans necessari després de l'exercici). El planificador controla que el pla generat no superi els 90 minuts diaris. 
\end{description}

\section{Domini}

\subsection{Tipus de dades}

En aquest domini s'han definit els següents tipus de dades:
\begin{description}
	\item[exercice] Representa un exercici
	\item[day] Identifica un dia
\end{description}

\subsection{Predicats}

En el domini s'han definits predicats diferents en funció de quina extensió havien de complir, fins a l'extensió 3 hi ha els següents predicats:
\begin{description}
	\item[before] Relaciona dos dies i s'utilitza per saber que un dia està abans que un altre (el dia 1 del pla es troba abans que el dia 2 per exemple).
	\item[precursor] Relaciona dos exercicis, el primer és precursor del segon, així és possible satisfer la relació d'un exercici precursor d'un altre. Si un exercici no té cap precursor, se li asigna el com a precursor l'exercici nul.
	\item[preparer] Relaciona dos exercicis, el primer és preparador del segon, així és possible satisfer la relació d'un exercici preparador d'un altre.
	\item[currentDay] identifica el dia actual, és necessari per realitzar tasques com canviar de dia o per saber els exercicis realitzats en un dia.
	\item[exerciceToday] Predicat on s'afegeixen els exercicis realitzats durant un dia en concret, al canviar de dia els exercicis marcats s'han d'eliminar per tornar-los a marcar, si cal, al següent dia. És útil satisfer la relació dels exercicis preparadors.
	\item[lastExerciceToday] S'utilitza per marcar quin ha estat l'últim exercici realitzat en aquell moment durant el dia, és útil per saber complir la relació del exercicis precursors. 
	\item[null] S'utilitza per marcar quin es l'exercici nul.
\end{description}

\subsection{Funcions}

Com que s'ha utilitzat \verb|fluent| (necessari per realitzar l'extensió 4). També s'han definit les següents funcions:
\begin{description}
	\item[exLevel] Relaciona un exercici amb el seu nivell
	\item[numExToday] S'utilitza per contar el nombre d'exercicis realitzats durant un dia i així satisfer l'extensió 3 (només utilitzat fins a l'extensió 3).
	\item[exTime] Relaciona un exercici amb la seva duració (només s'utilitza a l'extensió 4).
	\item[timeDay] S'utilitza per contar els minuts totals d'exercicis realitzats i així poder satisfer l'extensió 4 (només s'utilitza a l'extensió 4).
\end{description}

\section{Operadors}

Els operadors escollits han estat dos, un que és el que realitza l'exercici i l'altre que és el que canvia el dia. L'ordre en el que s'executin els operadors coincidira amb l'horari final.

\subsection{\texttt{do-exercice}}

La funció d'aquest operador és realitzar un exercici si s'han complert totes les precondicions d'aquest.

\textbf{Precondicions: } 
\begin{itemize}
	\item L'exercici encara no s'ha realitzat avui
	\item Si té un exercici precursor aquest és l'últim que s'ha fet avui
	\item Tots els exercicis preparadors ja s'ha fet avui
	\item (només extensió 3) El nombre d'exercicis fets avui és inferior a 6
	\item (només extensió 4) Els minuts d'exercicis fets avui més la duració de l'exercici són inferiors o iguals a 90.
\end{itemize}

Un cop s'ha realitzat l'exercici es posa aquest exercici com a l'últim realitzat avui, també s'afegeix a la llista d'exercicis realitzats avui, s'incrementa el nivell de l'exercici, (a l'extensió 3) s'incrementa el nombre d'exercicis realitzats i (a l'extensió 4) s'incrementa la duració de l'exercici.

\subsection{\texttt{skip-day}}

La funció d'aquest operador és canviar de dia per poder continuar en la creació del pla d'entrenament. Rep dos dies com a paràmetres i incrementa el dia del dia \verb|d1| al \verb|d2|.

\textbf{Precondicions: }
\begin{itemize}
	\item El dia \verb|d1| és el dia actual en el que s'està treballant
	\item El dia \verb|d2| ve després del dia \verb|d1|.
\end{itemize}

L'efecte de l'operador és marcar tots els exercicis com a no realitzats avui i també marcar que no hi ha cap exercici com a "últim exercici realitzat". A l'extensió 3 marca el nombre d'exercicis realitzats a 0 i a l'extensió 4 posa el nombre de minuts realitzats avui a 0.

\section{Jocs de prova}

\subsection{Jocs de prova 1 i 2}

\subsubsection{Introducció}

L'objectiu d'aquest joc de proves és demostrar que els horaris resultants compleixen totes les restriccions de l'extensió 3. El problema consisteix en un horari de 3 setmanes (15 dies) i 7 exercicis. L'objectiu és pujar l'exercici exA del nivell 5 al 7. L'exercici exA té com a preparadors els exercicis exP, exP2, exP3 i exP5 i com a precursor l'exercici exPr. A més, el exP2 té com a precursor el exP5. En el segon joc de prova només afegim exP4 com a preparador.

\subsubsection{Resultat esperat}

Com només cal pujar dos nivells, esperem que el planificador només indiqui dos dies d'exercici, saltant la resta sense fer res. Per complir les restriccions esperem que primer es faran els exercicis preparadors de tal manera que exP5 estigui abans que exP2, després s'hauria de fer exPr i finalment exA. Podem veure que si afegim el exP4 com a preparador superem el límit de sis exercicis al dia i ens hauria de dir que no existeix cap solució.

\subsubsection{Entrada PDDL}
\begin{lstlisting}
(define (problem exercices-example1)
	(:domain exercices)
	(:objects
		exP exA exPr exP2 exP3 exP4 exP5 none - exercice
		d1 d2 d3 d4 d5 d6 d7 d8 d9 d10 d11 d12 d13 d14 d15 - day
	)
	(:init
		(currentDay d1)
		
		(before d1 d2)
		(before d2 d3)
		(before d3 d4)
		(before d4 d5)
		(before d5 d6)
		(before d6 d7)
		(before d7 d8)
		(before d8 d9)
		(before d9 d10)
		(before d10 d11)
		(before d11 d12)
		(before d12 d13)
		(before d13 d14)
		(before d14 d15)

		(= (exLevel exA) 5)
		(= (exLevel exP) 1)
		(= (exLevel exP2) 1)
		(= (exLevel exP3) 1)
		(= (exLevel exP4) 1)
		(= (exLevel exP5) 1)
		(= (exLevel exPr) 1)

		(= (numExToday) 0)

		(null none)

		(precursor none exP)
		(precursor exPr exA) 
		(precursor none exPr)
		(precursor exP5 exP2)
		(precursor none exP3)
		(precursor none exP4)
		(precursor none exP5)
		
		(preparer exP exA)
		(preparer exP2 exA)
		(preparer exP3 exA)
		;(preparer exP4 exA)
		(preparer exP5 exA)
	) 
	(:goal (and (currentDay d15) (= (exLevel exA) 7)))
)
\end{lstlisting}

\subsubsection{Sortida PDDL}

	Sense l'exercici exP4:\newline

	\setlength{\parskip}{0em}
    0: SKIP-DAY D1 D2
    
    1: SKIP-DAY D2 D3
    
    2: SKIP-DAY D3 D4
    
    3: SKIP-DAY D4 D5
    
    4: SKIP-DAY D5 D6 
    
    5: SKIP-DAY D6 D7
    
    6: SKIP-DAY D7 D8
    
    7: SKIP-DAY D8 D9
    
    8: SKIP-DAY D9 D10
    
    9: SKIP-DAY D10 D11
    
    10: SKIP-DAY D11 D12
    
    11: SKIP-DAY D12 D13
    
    12: SKIP-DAY D13 D14
    
    13: DO-EXERCICE EXP NONE
    
    14: DO-EXERCICE EXP3 NONE
    
    15: DO-EXERCICE EXP5 NONE
    
    16: DO-EXERCICE EXP2 EXP5
    
    17: DO-EXERCICE EXPR NONE
    
    18: DO-EXERCICE EXA EXPR
    
    19: SKIP-DAY D14 D15
    
    20: DO-EXERCICE EXP NONE
    
    21: DO-EXERCICE EXP3 NONE
    
    22: DO-EXERCICE EXP5 NONE
    
    23: DO-EXERCICE EXP2 EXP5
    
    24: DO-EXERCICE EXPR NONE
    
    25: DO-EXERCICE EXA EXPR
    
 \setlength{\parskip}{0.7em}

	Amb l'exercici exP4:
	
	best first search space empty! problem proven unsolvable.
	
\subsection{Jocs de prova 3 i 4}
\subsubsection{Introducció}

L'objectiu d'aquest segon joc de proves és provar els resultats del domini que hauria de complir les restriccions de l'estensió 4. El problema manté moltes semblances amb el que hem vist anteriorment, ja que manté l'idea d'un sol exercici objectiu amb altres exercicis de precursors o preparadors. La diferència principal en aquest cas, però, és l'inclusió dels temps de cada exercici, cosa que s'ha dut a terme utilitzant fluents. També tenim una altra funció que ens indica el temps que portem sumat en el dia actual, que ens servirà per limitar el nombre d'exercicis al dia. El que utilitzarem per fer que les restriccions del problema no es compleixin aquest cop seràn els temps dels exercicis, que en aquets cas sumen 90 exactament.


\subsubsection{Resultat esperat}

De forma similar al test anterior, l'exercici exA és l'objectiu, encara que aquest cop haurà de pujar 5 nivells, i per tant es necessitaran 5 dies per dur-ho a terme. L'ordre dels exercicis preparadors i precursors haurà de mantenir les mateixes restriccions, amb l'afegit que ara també tenim l'exP4 que podrà anar en qualsevol posició, mentre sigui abans que exA. Haurem de tenir 7 exercicis per dia, però en aquesta extensió això no ens afecta. Si sumem qualsevol quantitat, per petita que sigui, a qualsevol dels temps dels exercicis, no hi haurà solució al problema ja que per les condicions descrites, han de dur-se a terme tots el mateix dia, i passaran dels 90 minuts.


\subsubsection{Entrada PDDL}
\begin{lstlisting}

(define (problem exercices-example3)
	(:domain exercices)
	(:objects
		exP exA exPr exP2 exP3 exP4 exP5 none - exercice
		d1 d2 d3 d4 d5 d6 d7 d8 d9 d10 d11 d12 d13 d14 d15 - day
	)
	(:init
		(currentDay d1)

		(before d1 d2)
		(before d2 d3)
		(before d3 d4)
		(before d4 d5)
		(before d5 d6)
		(before d6 d7)
		(before d7 d8)
		(before d8 d9)
		(before d9 d10)
		(before d10 d11)
		(before d11 d12)
		(before d12 d13)
		(before d13 d14)
		(before d14 d15)

		(= (exLevel exA) 5)
		(= (exLevel exP) 1)
		(= (exLevel exP2) 1)
		(= (exLevel exP3) 1)
		(= (exLevel exP4) 1)
		(= (exLevel exP5) 1)
		(= (exLevel exPr) 1)

		(= (exTime exA) 10)
		(= (exTime exP) 20)
		(= (exTime exP2) 10)
		(= (exTime exP3) 10)
		(= (exTime exP4) 15)
		(= (exTime exP5) 10)
		(= (exTime exPr) 15)

		(= (timeDay) 0)

		(null none)

		(precursor none exP)
		(precursor exPr exA) 
		(precursor none exPr)
		(precursor exP5 exP2)
		(precursor none exP3)
		(precursor none exP4)
		(precursor none exP5) 

		(preparer exP exA)
		(preparer exP2 exA)
		(preparer exP3 exA)
		(preparer exP4 exA)
		(preparer exP5 exA)
	) 
	(:goal (and (currentDay d15) (= (exLevel exA) 10)))
)
\end{lstlisting}

\newpage

\subsubsection{Sortida PDDL}
\setlength{\parskip}{0em}
    0: SKIP-DAY D1 D2
    
    1: SKIP-DAY D2 D3
    
    2: SKIP-DAY D3 D4
    
    3: SKIP-DAY D4 D5
    
    4: SKIP-DAY D5 D6
    
    5: SKIP-DAY D6 D7
    
    6: SKIP-DAY D7 D8
    
    7: SKIP-DAY D8 D9
    
    8: SKIP-DAY D9 D10
    
    9: SKIP-DAY D10 D11
    
    10: DO-EXERCICE EXP NONE
    
    11: DO-EXERCICE EXP3 NONE
    
    12: DO-EXERCICE EXP4 NONE
    
    13: DO-EXERCICE EXP5 NONE
    
    14: DO-EXERCICE EXP2 EXP5
    
    15: DO-EXERCICE EXPR NONE
    
    16: DO-EXERCICE EXA EXPR
    
    17: SKIP-DAY D11 D12
    
    18: DO-EXERCICE EXP NONE
    
    19: DO-EXERCICE EXP3 NONE
    
    20: DO-EXERCICE EXP4 NONE
    
    21: DO-EXERCICE EXP5 NONE
    
    22: DO-EXERCICE EXP2 EXP5
    
    23: DO-EXERCICE EXPR NONE
    
    24: DO-EXERCICE EXA EXPR
    
    25: SKIP-DAY D12 D13
    
    26: DO-EXERCICE EXP NONE
    
    27: DO-EXERCICE EXP3 NONE
    
    28: DO-EXERCICE EXP4 NONE
    
    29: DO-EXERCICE EXP5 NONE
    
    30: DO-EXERCICE EXP2 EXP5
    
    31: DO-EXERCICE EXPR NONE
    
    32: DO-EXERCICE EXA EXPR
    
    33: SKIP-DAY D13 D14
    
    34: DO-EXERCICE EXP NONE
    
    35: DO-EXERCICE EXP3 NONE
    
    36: DO-EXERCICE EXP4 NONE
    
    37: DO-EXERCICE EXP5 NONE
    
    38: DO-EXERCICE EXP2 EXP5
    
    39: DO-EXERCICE EXPR NONE
    
    40: DO-EXERCICE EXA EXPR
    
    41: SKIP-DAY D14 D15
    
    42: DO-EXERCICE EXP NONE
    
    43: DO-EXERCICE EXP3 NONE
    
    44: DO-EXERCICE EXP4 NONE
    
    45: DO-EXERCICE EXP5 NONE
    
    46: DO-EXERCICE EXP2 EXP5
    
    47: DO-EXERCICE EXPR NONE
    
    48: DO-EXERCICE EXA EXPR

\setlength{\parskip}{0.7em}


Si augmentem el temps d'algun exercici:

best first search space empty! problem proven unsolvable.

\end{document}